% Created 2023-08-05 Sat 21:22
% Intended LaTeX compiler: pdflatex
\documentclass[11pt]{article}
\usepackage[utf8]{inputenc}
\usepackage[T1]{fontenc}
\usepackage{graphicx}
\usepackage{grffile}
\usepackage{longtable}
\usepackage{wrapfig}
\usepackage{rotating}
\usepackage[normalem]{ulem}
\usepackage{amsmath}
\usepackage{textcomp}
\usepackage{amssymb}
\usepackage{capt-of}
\usepackage{hyperref}
\author{Cthulhu}
\date{\today}
\title{}
\hypersetup{
 pdfauthor={Cthulhu},
 pdftitle={},
 pdfkeywords={},
 pdfsubject={},
 pdfcreator={Emacs 27.1 (Org mode 9.3)}, 
 pdflang={English}}
\begin{document}

\tableofcontents

\section{Phase-2 Report: A Detailed Technical Progress Overview}
\label{sec:org9955a15}

\subsection{Introduction}
\label{sec:org9df2788}

Phase-2 of the project has built upon the solid foundation laid in Phase-1, with
a focus on expanding functionalities, refining user experience, and enhancing
security measures. This stage was marked by iterative development, rigorous
testing, and effective collaboration among the team members. 

\subsection{Frontend Development}
\label{sec:org056e427}

\subsubsection{Features and Functionalities}
\label{sec:orgf1f15f2}
\begin{itemize}
\item Gallery Component Development: The Gallery component was designed and
developed to display individual ArtworkCards, providing an immersive
browsing experience for the users.
\item ArtworkCard and ArtworkDetail Components Development: The representation
of individual artworks and their detailed view was implemented, allowing
users to explore and appreciate the artworks in depth.
\item UserAccount Component Development: The UserAccount component was developed
for managing user profile details and uploaded artworks, offering users a
personalized and manageable space.
\item Transition to Material-UI: To ensure a more consistent and robust UI, the
components were refactored to use Material-UI instead of Fluent UI.
\end{itemize}

\subsubsection{Design Components and Wireframes}
\label{sec:org521d13a}
The wireframes were updated to reflect the new components and changes in the
user interface. The components were further refined to ensure a cohesive and
intuitive user experience. 

\subsection{Backend Development}
\label{sec:org3306480}

\subsubsection{Features and Functionalities}
\label{sec:org9366fd4}
\begin{itemize}
\item Enhanced User Authentication: The user authentication process was improved
with the addition of a GET endpoint for the `/register` route, providing a
graphical way for users to sign up.
\item Artwork Model: An Artwork model was implemented to manage the artworks in
the database, facilitating efficient data handling.
\item VerifyToken Middleware: A middleware function was developed to verify
tokens, enhancing the security of the application by ensuring only
authenticated users can access certain routes.
\end{itemize}

\subsubsection{Secure Design Considerations}
\label{sec:org02518a0}
\begin{itemize}
\item Token Verification: The VerifyToken middleware was used to verify the
`auth-token` in the headers of requests, ensuring secure access to certain
endpoints.
\item Secure Routing: Routes were implemented for all the client components,
ensuring a secure and seamless navigation experience for the users.
\end{itemize}

\subsection{Integration of Client and Server}
\label{sec:org3b40aac}
The backend and frontend were integrated by connecting the API with the frontend
components. The user flow was streamlined, providing a smooth and intuitive
experience from registration to browsing the gallery. 

\subsection{Test-Driven Development (TDD)}
\label{sec:orgc219fc2}
The testing suite was expanded to cover the new functionalities:
\begin{itemize}
\item Unit Tests: Additional unit tests were written for the new functions and
methods.
\item Integration Tests: New integration tests were developed to test the
interactions between the new components and the backend.
\item End-to-End Tests: The end-to-end tests were updated to cover the new user
flow and functionalities.
\end{itemize}

\subsection{Changes and Adjustments}
\label{sec:org45d9a6e}
As the project evolved, several changes and adjustments were made:
\begin{itemize}
\item Transition to Material-UI: The decision to switch from Fluent UI to
Material-UI was made to ensure a more consistent and robust user
interface.
\item Routing Adjustments: The routing was adjusted to reflect the new user flow
and to ensure secure access to certain routes.
\end{itemize}

\subsection{Challenges and Overcoming Them}
\label{sec:org2fff939}
Phase-2 presented its own set of challenges:
\begin{itemize}
\item UI Library Transition: The transition from Fluent UI to Material-UI
required a thorough understanding of the new library and careful
refactoring of the components.
\item Secure Routing: Implementing secure routing required a nuanced approach
and careful testing to ensure only authenticated users can access certain
routes.
\end{itemize}

\subsection{Lessons Learned}
\label{sec:orga536c54}
Key takeaways from Phase-2 include:
\begin{itemize}
\item The Importance of Flexibility: The ability to adapt to changes and new
requirements played a crucial role in the successful completion of this
phase.
\item Value of Security: A consistent focus on security ensured a product that
not only delivers in terms of functionality and design but also in terms
of user privacy and data integrity.
\end{itemize}

\subsection{Conclusion}
\label{sec:orgfc35f79}

Phase-2 of the project has seen the successful expansion of functionalities,
refinement of user experience, and enhancement of security measures. The team's
commitment to best practices, continuous learning, and collaboration has led to
the successful completion of this phase and set the stage for the final phase of
the project. The lessons learned will continue to guide the development as we
move forward, with a focus on delivering a product that excels in terms of
quality, functionality, and design. 
\end{document}
